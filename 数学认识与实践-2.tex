%XeLaTeX
%\documentclass[twoside]{cctart}
\documentclass[twoside]{ctexart}
\usepackage{headrule,amsmath,vatola,amssymb,niceframe}
\usepackage{graphicx,multirow,bm,rotating}
\usepackage{booktabs,dcolumn}%关于小数点对齐
\newcolumntype{z}[1]{D{.}{.}{#1}}%关于小数点对齐
\usepackage{tabularx}%关于表自动折行
\usepackage{slashbox}%表格加斜线
\allowdisplaybreaks%%%%%%长公式自动分页

%=========================Page Format (此部分定义请勿修改)================================
\setlength{\voffset}{-3mm} \headsep 0.3 true cm \topmargin 0pt
\oddsidemargin 0pt \footskip 2mm \evensidemargin 0pt \textheight
21.5 true cm \textwidth 14.5 true cm \setcounter{page}{1}
\parindent 2\ccwd
\nofiles

\TagsOnRight\baselineskip 12pt

\catcode`@=11 \long\def\@makefntext#1{\parindent 1em\noindent
\hbox to 0pt{\hss$^{}$}#1} \catcode`\@=12

\catcode`@=11
\def\evenhead{}
\def\oddhead{}
\headheight=8truemm
% \footheight=0pt
\def\@evenhead{\pushziti
  \vbox{\hbox to\textwidth{\rlap{\rm\thepage}\hfil{\evenhead}\llap{}}
    \protect\vspace{2truemm}\relax
    \hrule depth0pt height0.15truemm width\textwidth
  }\popziti}
\def\@oddhead{\pushziti
  \vbox{\hbox to\textwidth{\rlap{}{\oddhead}\hfil\llap{\rm\thepage}}
    \protect\vspace{2truemm}\relax
    \hrule depth0pt height0.15truemm width\textwidth
  }\popziti}
\def\@evenfoot{}
\def\@oddfoot{}
\catcode`@=12
\allowdisplaybreaks%%%%%%长公式自动分页

\renewcommand{\baselinestretch}{1.2}
\def\D{\displaystyle} \def\n{\noindent}
\def\ST{\songti\rm\relax}
\def\HT{\heiti\bf\relax}
\def\FS{\fangsong\relax}
\def\KS{\kaishu\relax}
\def\sz{\small\zihao{-5}}
\def\vs{\vspace{0.3cm}}
\def\TT{{\rm T}}
\def\dd{{\rm d}}
\def\ee{{\rm e}}


\def\SEC#1#2#3{\vspace*{.2in} \begin{center}
{\bf\LARGE\zihao{3} #1}\\[.2in]
\zihao{4}\fangsong#2\\[.1in]
\small\zihao{-5}#3 \end{center}}

\def\ESEC#1#2#3{\vskip.2in \begin{center}
{\bf\Large #1}\\[.2in]
\normalsize #2\\[.1in]\end{center}
\footnotesize #3 }

\def\SUB#1{\vspace{.2in} \leftline{\large\bf\heiti\zihao{-4}#1}
\vspace{.1in}}
\def\sub#1{\leftline{\bf\heiti\zihao{5}#1}}

\def\REFERENCE{\vspace*{.2in}
{\noindent\bf\heiti\zihao{5}参考文献} \vspace*{.1in}}
%=========================================================================

\begin{document}

%--------------------------页眉-------------------------------------
\def\evenhead{{\protect\small{\zihao{-5}\songti \hfill 数~~学~~的~~实~~践~~与~~认~~识}
\hfill{\zihao{-5}\songti }\,\,XX\,{\zihao{-5}\songti 卷}}}
\def\oddhead{{\protect{\zihao{-5}\songti }\small\,\,XX\,{\zihao{-5}\songti 期}
\hfill {\small\zihao{-5}\songti{第一作者名, 等:  文章题目

  }
 }\hfill}}

%---------------------------首页页眉---------------------------------
\vspace*{-20mm} \thispagestyle{empty} \noindent \hbox to
\textwidth{\small{\zihao{-5}\songti 第}\,\,43\,{\zihao{-5}\songti
卷 第}\,\,XX \,{\zihao{-5}\songti 期}\hfill {\zihao{5}
数学的实践与认识}\hfill Vol.XX, No.\,XX } \vskip -0.2mm
\par\noindent
\hbox to \textwidth{\small 20XX~{\zihao{-5}\songti 年}\,\,XX
{\zihao{-5}\songti 月}\hfill MATHEMATICS IN PRACTICE AND THEORY
\hfill XXXX, 20XX} \vskip -1mm
\par\noindent
\rule[2mm]{\textwidth}{0.5pt}\hspace*{-\textwidth}\rule[1.5mm]{\textwidth}{0.5pt}

%%%%%%%%%%%%%%%%%%%%%%%%%%%%%%%%%%%%%%%%%%%%%%%%%%%%%%%%%%%%%%%%%
%\ziju{0.025} \noindent {\small\HT 文章编号:\ }1000-6788(2009)00-0000-00
%下面是方框图案的排法
%

%-----------------------脚注--------------------------------------------------
\footnotetext{{\HT 收稿日期:}\ 2021-10-24} \footnotetext{{\HT
资助项目:重点研究基地项目()}


}

%---------------------题目、作者、单位、摘要、关键词, 中图分类号(必须提供)------------------
\vspace{0.5cm}
%{中文题目}{作者}{单位}
\SEC{能源}{张三}{理工大学经济与管理学院}

\vspace{0.4cm} \vskip.1in {\narrower\zihao{5}\fangsong\noindent
{\heiti 摘\quad 要:}\ \ %下面放中文摘要




\vskip.1in \noindent {\heiti 关键词:}\quad %下面放中文关键词



}


\normalsize \normalsize \abovedisplayskip=2.0pt plus 2.0pt minus
2.0pt \belowdisplayskip=2.0pt plus 2.0pt minus 2.0pt \baselineskip
16pt


%%%%%%%%%%%%%%%%%%%%%%%%%%%%排版的基本要求%%%%%%%%%%%%%%%%

% 1. \sub{1.1~~}二级标题命令 \sub{1.1.1~~}  %{\HT 定理~1}\quad   {\HT 证明}\quad

% 2. 在排公式号时,请用顺序号如:(1),(2),(3) 公式号的命令用\tag 1,  文中的句号用``."表示,

% 3. 在公式中的括号要随公式中的分式的大小来变化.
%   如果公式中有 \frac \int \sum等, 括号要用\left \right来定义括号.

% 3. 所有的标点符号都用"英文"中的标点符号----就是"半角"加空格
% 4. 图中的文字一定要清晰, 不要用彩色图.
% 5. 参考文献的顺序应按照在文中出现的先后顺序来排序.

%%%%%%%%%%%%%%%%%%%%%%正文%%%%%%%%%%%%%%%%%%%%%%%%%%%%%%%%%

\SUB{1\ \ 引言}%一级标题







%%%%%%%%%%%%%%%%%%%%%参考文献%%%%%%%%%%%%%%%%%%%%%%%%%%%%%%%%%%%%%%%%

\REFERENCE

\baselineskip 11pt {\small \parindent=4.75mm

%注意:所英文文献作者名应按照"姓"在前"名"("名"要缩写)在后书写.

\REF{[1]}




      }





%%%%%%%%%%%%%%%英文部分%%%%%%%%%%%%%%%%%%%%%%%
\vspace{0.4cm}
%{英文题目}{作单}{单位}
\ESEC{Oil Price}{San Zhang}{ University of Technolgoy}


\vspace{0.4cm}
 \vskip.1in \rm \noindent {\narrower\small {\bf Abstract:}\ \ %放英文摘要


\vskip.1in \noindent{\bf Keywords:}\ \ %放英文关键词

}






\end{document}
